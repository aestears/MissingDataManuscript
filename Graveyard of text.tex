
 
 %Old Intro Outline
 % Older version
% I. Missing data are a common problem in science and can arise in a variety of ways (MCAR, MAR, and MNAR)
% II. Many approaches exist for dealing with missing data from simple deletion/single imputation to more complex, model-based approaches (1-2 paragraphs)
% III. Missing data in ecological settings can present unique challenges.
%   1. Ecological data are often structured as autoregressive time series in which current data partially depend on the value of previous data. In such settings, missing values can be especially problematic as they affect inference on multiple data points.
%   2. Additionally, ecological data can often correspond to non-normal distributions (e.g., count data) that have not been rigorously explored with existing methods of dealing with missing data.
% IV. Here we ....
 % Old Intro Text
 %Missing data are a frequent concern in science \citep{rubin_inference_1976, mcknight_missing_2007,allison_missing_2002,little_statistical_2002}, presenting a common challenge in different real-world datasets, including experimental, observational, and meta-analytical data. The problem of missing data is particularly pervasive in primarily field-based disciplines such as ecology, where researchers face many unpredictable environmental barriers to complete data collection\citep{lopucki2022handling, nakagawa_missing_2008}. Datasets with missing observations can arise in many ways, including incomplete data collection due to faulty sensors, inaccessibility of field sites, or even a global pandemic; as well as human error during the data entry or collection processes that necessitates removing observations due to inaccuracy. The consequences of using data with missing values in subsequent analysis include reduced statistical power \citep{kang2013prevention, moritz_imputets_2017} and biased estimation of parameters, leading to both inaccurate and imprecise conclusions \citep{aleryani2018dealing, kim_transcending_2018, junger_imputation_2015}.% Additionally, missing data typically increase uncertainty, which should be adequately accounted for when confronting models with data \citep{nakagawa_missing_2008,graham_missing_2009}. %AES not really sure what the previous sentence is adding here
 
%Missing data is an acknowledged problem both within and outside ecological disciplines, yet it is relatively rare that researchers adjust their analyses in statistically rigorous or apposite ways to account for missing observations \citep{nakagawa_missing_2015, junger_imputation_2015, velicer2005comparison}. 

%Per \citet{luengo2010study}, missing $<$1\% of the data is trivial, missing 1-5\% is manageable, and missing $>$5-15\% of the data requires sophisticated handling techniques because it can severely diminish model accuracy \citep{luengo2010study}. Further, data can be Missing Completely at Random (MCAR), Missing at Random (MAR), or Missing Not at Random (MNAR) \citep[][; see Fig.\ \ref{fig:missingtypes}]{rubin_inference_1976}. The MCAR category assumes that the pattern of missing data is independent of the observed and unobserved data \citep{ellington_using_2015, horton2007much}, a rarely tested or valid assumption for empirical ecological datasets due to observer error, collinearity, autocorrelation, and natural variation \citep{nakagawa_missing_2015, hossie_confronting_2021, little_statistical_2002}. The MAR category assumes that missing data are associated with other observed variables but independent of any unobserved or missing variables \citep{newman_missing_2014, ellington_using_2015}. The MNAR category applies to data when the probability of missingness depends on the value of the missing (or unobserved) data. For example, MNAR data occur in ecological time series when sensor accuracy and function are lost during extreme conditions, such as high streamflow, high wind speeds, or low temperatures. Data can be MNAR even if an unobserved variable causes missingness, for example, if streamflow is not measured but high flows cause missingness in another measured variable of interest  \citep{nakagawa_missing_2015}. MNAR assumptions must be specified by the researcher and included during analysis to avoid biased parameter estimates \citep{newman_missing_2014, dong2013principled}. 

%These different sources of missingness in datasets lead to different "types" or "mechanisms" of missing data: data are missing completely at random ("MCAR") when missing observations occur in a random pattern in the space or time across which the data were collected, and the occurrence of missing observations in one variable is unrelated to the occurrence of missing observations in another variable (e.g. missingness in the predictor is unrelated to missingness of the response variable); data are missing at random ("MAR") when missing observations occur in a random pattern (as in MCAR), but occurrence of missing observations of one variable is related to occurrence of missingness in another variable (e..g missingness of the predictor occurs at the same time steps as missingness of the response); and data are missing not at random ("MNAR") when the occurrence of missingness in a variable is related to the values of that variable (e.g. a value will not be measured by a streamflow sensor when flow is too high) \citep{nakagawa_missing_2015, gelman_data_2006}.

  
%Perhaps one reason that missing data are often dealt with in a statistically inappropriate manner in ecological analyses is that there are multiple existing tools and workflows for addressing missing data, and their suitability for a given use depends on multiple factors such as the type and amount of missingness, the type of data, and nature of the ultimate analysis \citep{nakagawa_model_2011, hossie_confronting_2021}. %Very small amounts of missing data can be overlooked, while larger amounts must be addressed to preserve model accuracy. %% I don't think we need this sentence, since it's sort of undermining our point...


% III. Methods have been proposed for dealing with missing data. However, many were formulated without considering the challenges of missing data in autoregressive time series. While there are some methods that have been developed specifically to address missingness in autoregressive time series, we do not have clear information about how these various methods perform with different types and amounts of missingness and with different types of time series data. 

% IV. Here, we evaluate the precision and accuracy of different statistical approaches for dealing with missing data in autoregressive models of ecological time series data. We test these missing data approaches on real and simulated datasets with both Gaussian and Poisson error distributions. We artificially introduce missingness of different types and increasing amounts into all of these time series, and quantify the performance of different missing data approaches across these three axes of variation in the time series data used (error distribution of the time series, amount of missingness, and type of missingness). This comparison of approaches for dealing with missing data in ecological time series provides a novel comparison of multiple previously-proposed methods of dealing with missing data, some of which have not been previously used with time series data, and will help ecologists determine which method is best suited for addressing the problem of missing data in their particular time series dataset. 


% p2: Many approaches exist for dealing with missing data from simple deletion/single imputation to more complex, model-based approaches (1-2 paragraphs)

%Many methods have been developed to manage missing data and mitigate their impact on statistical analyses (Fig. \ref{fig:ConceptualFigure}). Broadly speaking, these methods can be categorized into simple data processing approaches that occur prior to full statistical analysis or model-based approaches that directly incorporate the missing data into the model-fitting process. The former category primarily consists of data deletion approaches and single imputation. A common example of a data deletion approach is complete case analysis, which, as its name implies, only analyzes complete cases of data while ignoring or omitting cases that contain one or more missing values. However, this approach reduces statistical power and biases parameter estimates \citep{nakagawa_model_2011, aleryani2018dealing}. Rather than removing missing data, single imputation `fills in’ missing data to create complete cases \citep{nakagawa_model_2011} by replacing missing values with either the mean, median, or mode of the data, depending on the data type \citep{kang2013prevention, nakagawa_missing_2015, onkelinx_working_2017}. Although it helps to maximize input, single imputation has received wide criticism due to its failure to consider uncertainty in the imputed values, resulting in overconfident precision \citep{fichman2003multiple, nakagawa_model_2011, aleryani2018dealing}. In contrast to these relatively simplistic methods, several model-based approaches have been developed to directly account for the missing data in the model fitting process. For example, an extension of single imputation, multiple imputation (MI), imputes missing data points many times, resulting in multiple new datasets (\textit{m} datasets) where the imputed values vary stochastically for each missing data point, thus allowing uncertainty to be calculated for parameter estimates \citep{rubin1996multiple, rubin1988overview, nakagawa_model_2011, nakagawa_missing_2015}. In contrast to MI, some model-based approaches use iterative algorithms to handle missing data, either by alternating between parameter estimation and data imputation steps (expectation maximization) \citep{nadjafi2022expectation,li2019expectation, kang2013prevention} or by iterating between predicting and updating steps in time series data (Kalman filter) \citep{kalman_filter_1960}. One more recently developed model-based approach, data augmentation (DA), treats the missing data themselves as unknown parameters to be estimated along with the parameters of a specified model for the data-generating process (e.g., means or standard deviations) \citep{kong_sequential_1994}. While many methods exist for dealing with missing data, not all of them have been adopted for ecological datasets, nor have they been evaluated for their ability to handle missing data in ecological settings.


%Time series with non-Gaussian error distributions are common in ecology and often pose additional modeling challenges. For example, population count data may be modeled using a Poisson or negative binomial distribution, which precludes the use of classical ARIMA models. In these cases, many of the aforementioned model-based approaches for handling missing data must be adjusted to account for the discrete nature of the missing observations. Interestingly, this type of time series has not been confronted with the existing methods of addressing missing data. For example, when using a DA approach, discrete data require exploring discrete parameter space since the missing data are treated as parameters, something commonly-used tools may not be able to handle without additional thought and programming \citep[for example, STAN, a commonly-used software for fitting Bayesian models requires the user to marginalize over discrete parameters in their model specification;][]{carpenter_stan_2017}. While we avoid the technical details here, we aim to illustrate that, with some extra effort, such model-based approaches as EM and DA can be used for incomplete discrete-valued time series.

% p5: Here, we evaluate the precision and accuracy of different statistical approaches for dealing with missing data in autoregressive models of ecological time series data. We test these missing data approaches on real and simulated datasets with both Gaussian and Poisson error distributions. We artificially introduce missingness of different types and increasing amounts into all of these time series, and quantify the performance of different missing data approaches across these three axes of variation in the time series data used (error distribution of the time series, amount of missingness, and type of missingness). This comparison of approaches for dealing with missing data in ecological time series provides a novel comparison of multiple previously-proposed methods of dealing with missing data, some of which have not been previously used with time series data, and will help ecologists determine which method is best suited for addressing the problem of missing data in their particular time series dataset. 


% There is limited knowledge of how accurately methods such as multiple imputation or model-based approaches are able to recover parameter estimates or capture the uncertainty imposed by missing data, especially in the context of time series data. Here, we evaluate the ability of multiple methods to recover accurate and precise model parameter estimates and forecast predictions for different amounts and types of missing data. To address this question, we constructed two simple yet realistic time series models for empirical and simulated data, while sequentially removing increasing amounts of missing data and changing the mechanisms of missingness. We deliberately choose mechanisms of missingness that are common in ecological datasets, such as missing data caused by a sensor breaking when temperatures are at extremes or missing multiple successive years in data collection in a long-term population census study. We apply our approach to two illustrative time series models, demonstrating how these methods perform with continuous data containing Gaussian error versus discrete count data with non-Gaussian error. Ultimately, we intend for this comparative study to serve as a resource for ecologists and environmental scientists in search of robust, reproducible methods for confronting time series models with missing data.

 % end of Old Intro Text
 
 However, the imputation methods vary greatly in their ability to account for the uncertainty imposed with the imputation. There are many options for data imputation, which range in complexity and the ability of the method to account for uncertainty imposed by missing data imputation. Ignoring data gaps and single imputation (e.g., with a central tendency or interpolation) are common approaches in ecology, but they introduce bias and do not account for missing data uncertainty \citep{nakagawa_missing_2008, graham_missing_2009}. These approaches create unrealistically narrow standard errors \citep{nakagawa_missing_2015,mcknight_missing_2007} and introduce bias to parameter estimates if there is a non-random pattern to missingness. Furthermore, ignoring missing data is not feasible when considering auto-correlation in models where a complete data set is necessary. 


 \section*{Graveyard of orphaned text} %so sad
A common theme among missing data research is the idea that using list-wise deletion to ignore the issue of missing data is never a safe practice. Ignoring missing data will almost certainly impart bias on any statistical analyses by reducing the statistical power. [EXAMPLE OF HOW DELETING DATA AFFECTS INTERPRETATION OF RESULTS]. It is nearly always better to impute missing data points than to ignore them completely. However, imputation alone is only safe when the missing data mechanism is ignorable.  



Missing data are so prevalent that an entire sub-field of science is devoted to studying the issue. \cite{rubin_inference_1976} and colleagues developed much of the initial theory and operational definitions for missing data research. These definitions include classifying missing data into categories based on the missing data mechanism, which are based in probability theory and represent the statistical relationship between observations and the probability of missing data \citep{rubin_inference_1976}. When the missing data mechanism is controlled by chance (missing at random, MAR), or by a mechanism known to the researcher (missing completely at random, MCAR), the missingness mechanism can be ignored, which allows for imputing missing data with minimal bias \citep{little_statistical_2002, nakagawa_missing_2015}. Alternatively, when the missing data mechanism is unknown but may in fact have a relationship with present and missing observations (missing not at random; MNAR), the missing data mechanism must be built into the model, leading to a more complex imputation process. 


 MI has been used fairly commonly in ecology, with multiple studies evaluating methods and approaches to conduct MI for functional traits \citep{taugourdeau_filling_2014,johnson_handling_2021,penone_imputation_2014}, population biology \citep{onkelinx_working_2017}, time-series \citep{hui_gap-filling_2004}, and meta-analyses \citep{ellington_using_2015}. 


There are a myriad of approaches to account for auto-correlation in ecological models \cite{hefley_basis_2017, borcard_partialling_1992, padilla_space-time_2020}, yet approaches methods papers highlighting approaches to impute missing data generally do not consider the uncertainty imposed with imputation; rather there is a focus on recovering known data \cite{penone_imputation_2014, johnson_handling_2021, ellington_using_2015}.



Multiple imputation and data augmentation are two approaches for missing data imputation that can account for missing data uncertainty and introduce less bias than other common approaches in ecology, such as ignoring data gaps or single imputation \citep{nakagawa_missing_2008, graham_missing_2009}.

\section*{Outline of sections}
\subsection*{Comparing parameter estimates between Bayesian and Multiple Imputations}

\begin{itemize}
\item Simulate time-series data and test the how MI and DA fare at estimating different parameters
% use more than one time series process (AR, MA).  Probably should be only state space, since I see no reason not to use state space with ecological time series.
\begin{itemize}

\item TO RESOLVE: 1) How many different simulated datasets and which ones? How many are enough to make a point? %1 each pop and GPP plus simulated data

2) How many imputation datasets \textit{m} should I use? Maybe test small and large? Will need to send model to server farm to do larger ones..Could be interesting context for the paper (i.e., sure \textit{m}=200 works well but it takes forever).  %no idea
\end{itemize}
\item Test if MI is better than DA in terms of MI can include ALL variables, while DA can only use data in the model. This is widely seen as one of the only pitfalls of DA. %Yes, if missing X variables, MI seems better unless one build a model first to use V variables to predict other x variables.  But that only works when there are say >4 x variables.  But you are leaving the time series world in this case, we don't often have lots of x variables unless we are talking about climate, and they are usually not missing than to Prism.
\end{itemize}




 Different approaches to estimate missing data that account for the uncertainty of estimates:
Data Augmentation
  \begin{itemize}
 \item \cite{clark_hierarchical_2005} Bayesian approach for mark-recapture and life-history prediction. Not very accessible-but a solid example.
 \item \cite{worrall_dissolved_2020} They estimate missing DOC concentration and Q data from a broad sampling program at many sites across the UK (similar to the US USGS sampling program). They seem to estimate ALL data points using a GLM based on site, month, and year and known DOC concentration and Q data in a Bayesian hierarchical model. 
\item \cite{colchero_basta_2012} Use Bayesian approach to estimate missing birth and death data for mark-recapture calculations. They created R code package for this. There isn't a ton of information that would be helpful for estimating missing data, but it is a solid example. 
 \item \cite{ma_bayesian_2018}: Summarize approaches to estimate missing data with Bayesian approaches including ignorable and non-ignorable settings. Very dense. A lot of equations. Not very accessible. 
\item \cite{kong_sequential_1994}: Early paper highlighting the method. Very dated.
\item \cite{bertschinger_bayesian_2021}: Economics focused. Only paper I have seen to use Stan Bayesian parameter estimation for missing data in a time series. They have a much more complicated missing data code. 
\item \cite{chatzilena_contemporary_2019}: Uses Stan Bayesian parameter estimation for missing data estimate that are discrete. Not time-series. "Marginalize out the missing discrete data".
\item \cite{scutari_bayesian_2019}: Possibly helpful summary of Bayesian approach. Specifically B network analysis. Very dense though. Hard to sift through.
\item \cite{jackman_estimation_2000} Provide a solid explanation of Bayesian approaches for estimating missing data presented in the context of political science data examples.
   \end{itemize}

% Text from the old version of the introduction:
In addition, if data are autocorrelated, as is true in many time series, using a statistic of the entire dataset as a replacement for missing values ignores the greater influence that nearest neighbor values may have, or the influence of missing values on subsequent time-steps.

The MI approach is widely accepted among researchers in various fields (e.g., \citep{van2018flexible, nakagawa_missing_2015}) because it has minimal bias, is robust to normality assumptions, accommodates small sample sizes and a high proportion of missing data, and leads to valid statistical inference compared to other methods \citep{kang2013prevention, tran2017missing, van2006imputation}. Further, it is devoid of excessively optimistic standard error that is typical of other imputation methods, e.g. maximum likelihood estimates \citep{van2018flexible, newman_missing_2014}. However, Ellington et al. (\citeyear{ellington_using_2015}) show that while MI was robust to the type of data missingness (MAR, MCAR, MNAR), accuracy declined when the raw data used to compute the effect size (e.g., mean population density) and weighting parameters (e.g., standard error of population density) were missing. Since this study did not examine auto-regressive data, it is uncertain if the effects of missingness  might be compounded when attempting to apply an MI approach to autocorrelated time series. As such, MI is not commonly applied to time series data in ecology \citep{hossie_confronting_2021}, perhaps due to the large size of time series datasets and the increased computation time that multiple imputations require. 

Model-based approaches, including the Kalman filter algorithm and Expectation maximization, use algorithms to handle missing data. Commonly used in time-series approaches, the Kalman filter is a two-step algorithm that consists of a prediction (or forecasting) step and an updating step \citep{kalman_filter_1960}. First, an observation is estimated based on a previous observation (the prediction step), and then this estimate is compared to the true value of the observation to determine a weighted average of the two values (the updating step). When observations are missing, this approach estimates the true, but unobserved state of the system and cannot perform the updating step, so estimates are derived purely from predictions based on the previous timestep. Expectation maximization (EM) is another flexible approach we consider to be `model-based'. EM is also an iterative algorithm with two steps; the expectation and the maximization steps \citep{nadjafi2022expectation,li2019expectation}. In the expectation stage, missing data are ``filled-in" with their expected values under a specified model with a ``working estimate" of the parameters. The filled-in data are then used to update the parameter estimates with their maximum likelihood estimates given the filled-in data. This iterative process continues until estimates converge (see below for more details) \citep{kang2013prevention}. EM is sensitive to the initial parameter vector, which may affect algorithm convergence in the likelihood function \citep{sammaknejad2019review, dempster1977maximum}. A random disturbance term is generated for each imputed value to reflect the uncertainty associated with this imputation method \citep{kang2013prevention}. EM is widely employed for managing missing data across disciplines \citep{nadjafi2022expectation, vamanu2019altered, sammaknejad2019review}. However, the approach does not scale with sample size and takes a long time to converge, limiting its current use in empirical studies. 

More recently developed, model-based approaches present alternatives to handling data gaps based on probabilistic models. One such approach is data augmentation (DA), which treats missing observations as unknown parameters to be estimated along with the parameters of a specified model for the data-generating process (e.g., means or standard deviations). Markov chain Monte Carlo simulation can be used to simultaneously solve for parameters and missing data in a Bayesian framework. Outputs contain samples from posterior distributions for both the model parameters and missing data points, from which one can derive uncertainty estimates for both. While the researcher controls the number of samples in the posterior distribution, the number of values that each sample is allowed to explore is theoretically infinite. Thus, DA can be regarded as MI under a specified model, in which we can impute missing observations an infinite number of times \citep{kong_sequential_1994}. While DA may not improve upon the computation time necessary to run MI for complex data sets or models, it may substantially improve uncertainty estimates for the missing data. DA is a relatively new approach and, thus, has rarely been used in the ecological sciences. A few exceptions include estimating missing discharge and nutrient concentration data in time and space \citep{worrall_dissolved_2020}, and implementation of DA to impute missing mark-recapture data \citep{colchero_basta_2012, clark_hierarchical_2005}. Methodological studies have been conducted using DA for economic models \citep{bertschinger_bayesian_2021} and social sciences \citep{jackman_estimation_2000, honaker_what_2010}. 

Moreover, auto-correlation between observations is common in time series data \citep{lichstein_spatial_2002, dale_spatial_2002}, and ignoring the auto-correlation structure of data in models can impart error when attempting to relate environmental controls to a specific process \citep{fieberg_understanding_2012}, highlighting the unique nature of time series data.